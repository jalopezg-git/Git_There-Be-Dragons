% ==== ADDITIONAL STUFF ====
\section{Additional stuff}

\begin{frame}{Configuration (1/2)}
  \begin{itemize}
  \item \alert{480+} options. Git searches configuration at:
    \begin{description}[]
    \item[/etc/gitconfig] System-wide configuration.
    \item[\~{}/.gitconfig] User-specific configuration.
    \item[\$GIT\_DIR/config] Repository specific.
    \end{description}
  \item Can be edited manually or using \texttt{git-config(1)}, e.g.\\[-.7ex]
    {\ttfamily\tiny \$ git config [--system|--global|--local] user.email 'John Doe'}\\[1em]
    \tikz\node[TTfamily]
              {[user]\\
               email = jalopezg@inf.uc3m.es\\
               name = Javier López-Gómez\\
               \ldots{}};\\[1ex]
  \end{itemize}
\end{frame}

\begin{frame}{Configuration (2/2)}
  \texttt{alias.*} options may be used to create command aliases, e.g.\\[1ex]
  \tikz\node[TTfamily]
            {\$ git config --global alias.sb 'show-branch\ \ @\ \ @\{push\}'\\
              \$ git sb\\
              {\color{cred}!} [@] Updated README.md\newline
              \phantom{!}{\color{cgreen}!} [@\{push\}] Closes issue \#16\\
              --\\
              {\color{cred}+}\phantom{+} [@] \ldots{}};\\[2em]
  \alert{FYI, see the \texttt{git-config(1)} manual page.}
\end{frame}

\begin{frame}{Fsck and garbage collection}
  \begin{description}[]\itemsep=2em
  \item[git-fsck(1)] Verifies the connectivity and validity of the objects.\\
    \tikz\node[TTfamily]
              {\$ git fsck [--unreachable] [--no-reflogs] [--lost-found] [\ldots{}]};
  \item[git-gc(1)] Runs housekeeping tasks, e.g. pack objects/refs, remove unreachable objects, prune reflog, etc.\footnote{\texttt{git gc --auto} may automatically run as part of some git commands.}
    \tikz\node[TTfamily]
              {\$ git gc [--aggressive] [--auto] [\ldots{}]};
  \end{description}
\end{frame}

\begin{frame}{Hooks (1/2)}
  Hooks are programs that are executed at certain points, e.g. after a merge \emph{(post-merge)}, or before git-receive-pack updates refs \emph{(pre-receive)}.
  
  \begin{itemize}[<+->]
  \item Invoked locally/on the remote end\footnote{Stdout and stderr are forwarded.}
  \item Must be executable (+x)
  \item IN: environment, command-line arguments, stdin\\ OUT: stdout, stderr, exit status
  \item Can be used for commit validation, issue management or triggering a build (CI)
  \end{itemize}
\end{frame}

\begin{frame}{Hooks (2/2)}
  See templates installed into \texttt{.git/hooks/} and the \texttt{githooks(5)} manual page.\\[2.7em]
  \begin{columns}
    \ttfamily\scriptsize
    \column{.33\textwidth}
    applypatch-msg\\
    pre-applypatch\\
    post-applypatch\\
    pre-commit\\
    prepare-commit-msg\\
    commit-msg\\
    post-commit
    \column{.33\textwidth}
    pre-rebase\\
    post-checkout\\
    post-merge\\
    pre-push\\
    pre-receive\\
    update\\
    post-receive
    \column{.33\textwidth}
    post-update\\
    push-to-checkout\\
    pre-auto-gc\\
    post-rewrite\\
    sendemail-validate\\
    fsmonitor-watchman\\
    p4-pre-submit
  \end{columns}
\end{frame}

\begin{frame}{git-daemon, git-instaweb}
  ``git-daemon - A really simple server for Git repositories'' [\texttt{git-daemon(1)}], e.g.\footnote{It normally listens on port TCP 9418.}\\[1ex]
    \tikz\node[TTfamily]
              {[alice@earth \~{}]\$ git daemon --verbose --base-path=\$HOME/repos \textbackslash{}\ --reuseaddr --export-all \$HOME/repos/*/.git\\[2em]
              [bob@mars \~{}]\$ git clone git://earth/foo};\\[2em]

  git-instaweb allows browsing a repository\footnote{Requires perl-cgi and lighttpd.}, e.g.\\[1ex]
    \tikz\node[TTfamily]
              {\$ git instaweb [--local] --httpd=lighttpd --port=8080\\
              \$ git instaweb --stop};
\end{frame}

\begin{frame}{Other projects}
  \centering
  \tikz[every node/.style={font=\fontsize{28}{28}\selectfont,mLightBrown83}]\path
    node{git-annex} ++(-4em,-9em) node[opacity=.74]{git-crypt} ++(12em, 4em) node[opacity=.59]{libgit2};
\end{frame}
