% ==== [sudoer] MORE PORCELAIN ====
\section{\emph{[sudoer]} More porcelain}

\begin{frame}[fragile]{Save/Restore a dirty working directory}
  \texttt{git-stash(1)} saves the current state of the working directory + the index, and goes back to a clean WD.

  Saved changes can be restored with {\ttfamily\scriptsize\$ git stash pop}. Git-stash stack can be dumped by {\ttfamily\scriptsize\$ git stash list}.
  \pause

  \begin{lstlisting}[style=bash]
    $ echo foo > README.md
    $ git status
    On branch foo
    Changes not staged for commit:
    ``        `{\color{cred}modified:   README.md}\pause`

    $ git stash
    Saved working directory and index state WIP on foo: 9f7f586 README.md has been added
    $ git status
    On branch foo
    nothing to commit, working tree clean`\pause`

    $ git stash pop
    On branch foo
    Changes not staged for commit:
    ``        `{\color{cred}modified:   README.md}`
    Dropped refs/stash@{0} (35365e0c188e877ded1ecdd8190ec5bb1b6c2c1b)
  \end{lstlisting}
\end{frame}

\begin{frame}{Applying changes from other branches}
  \begin{columns}
    \column{.55\textwidth}
    \texttt{git-cherry-pick(1)} apply the changes introduced by the given commits, e.g.\par
    {\ttfamily\scriptsize\$ git cherry-pick 9f7f586}.\\[1em]
    The patch may not apply cleanly; if that is the case, you are required to resolve conflicts

    \column{.45\textwidth}
    \centering
    \tikz[every label/.style={font=\tiny}]{
      \graph[tree layout,grow'=up,nodes={simple commit obj,as=}]
            { a <- aa <- { b <- bb <- bbb[mLightBrown,label=left:9f7f586] <- bbbb[label=90:master],
                          c <- cc }
            };
      \draw<2>[<-] (cc) -- ++(0em,2.6em)
                   node(ccc)[simple commit obj,mLightBrown83,label=right:0c151e4]{};
      \foreach \i [count=\c] in {cc,ccc} \path<\c> node[above=0pt of \i,font=\tiny]{foo};
    }
  \end{columns}
\end{frame}

\begin{frame}{Rebase (+ interactive rebase!) (1/3)}
\end{frame}

\begin{frame}{Rebase (+ interactive rebase!) (2/3)}
\end{frame}

\begin{frame}{Rebase (+ interactive rebase!) (3/3)}
\end{frame}

\begin{frame}[fragile]{More about fixing history (1/2)}
  This is so common that \texttt{git-commit(1)} has the \texttt{--squash} and \texttt{--fixup} options. They mark commits to be automatically squashed. Rewriting occurs after a {\ttfamily\scriptsize\$ git rebase --autosquash}.
  \pause
  \begin{lstlisting}[style=bash]
    $ git log --oneline
    e7a2019 (HEAD -> master) Any other changes
    9f7f586 Added README.md
    02a7fb9 Added bar.txt`\pause`

    $ echo foo >> README.md && git commit -a --fixup 9f7f586
    $ git log --oneline
    24a54df (HEAD -> master) fixup! Added README.md
    e7a2019 Any other changes
    9f7f586 Added README.md
    02a7fb9 Added bar.txt`\pause`

    $ git rebase -i --autosquash 02a7fb9
    Successfully rebased and updated refs/heads/master.
    $ git log --oneline
    528efb7 (HEAD -> master) Any other changes
    a59735c Added README.md
    02a7fb9 Added bar.txt
  \end{lstlisting}

  \gitRebaseWarning
\end{frame}

\begin{frame}{More about fixing history (2/2)}
  If you only need to rewrite the last commit use\\ {\ttfamily\scriptsize\$ git commit --amend}

  \gitRebaseWarning
\end{frame}

\begin{frame}{git filter-branch}
  \Q I know how to rewrite commits. Can I automate the process?\\
  \pause
  \A \texttt{git-filter-branch(1)} lets you rewrite branches, applying filters to modify each tree/information about each commit, e.g. \\[1em]

  \tikz\node[TTfamily]
            {\$ git log --oneline\\
            {\color{cyellow}92cb761} (HEAD -\GT{} foo) Added nsswitch.conf\\
            {\color{cyellow}9f7f586} Added README.md\\
            {\color{cyellow}02a7fb9} (bar) Added bar.txt\\[1ex]
            \$ git filter-branch --msg-filter 'sed -e "s/Added \textbackslash([[:graph:]]*\textbackslash)\$/\textbackslash{}1 has been added/"' foo\\[1ex]
            \$ git log --oneline\\
            {\color{cyellow}6e9fbd6} (HEAD -\GT{} foo) nsswitch.conf has been added\\
            {\color{cyellow}63feb3c} README.md has been added\\
            {\color{cyellow}2fe54f3} bar.txt has been addded};

  \gitRebaseWarning
\end{frame}

\begin{frame}{Comparing branches}
  \Q Can I see where each of the given branches is w.r.t. others?\\
  \pause
  \A \texttt{git-show-branch} is your friend. Also, \texttt{git log --graph --oneline \ldots}\\[1em]

  \tikz\node[TTfamily]
            {\$ git show-branch master foo\\
            {\color{cred}!} [master] Added README.md\newline
            \phantom{!}{\color{cgreen}*} [foo] Added nsswitch.conf\newline
            \phantom{!*}{\color{cyellow}!} [bar] Added bar.txt\\
            ---\newline
            \phantom{+}{\color{cgreen}*}\phantom{+} [foo] Added nsswitch.conf\\
            {\color{cred}+}{\color{cgreen}*}\phantom{+} [master] Added README.md\\
            {\color{cred}+}{\color{cgreen}*}{\color{cyellow}+} [bar] Added bar.txt\\[1ex]
            \# To include all remote-tracking and local branches:\\
            \$ git show-branch --all};
\end{frame}

\begin{frame}{Share by other means}
  TAR or ZIP archives of a particular tree can be created by \texttt{git-archive(1)}, e.g.\\[1ex]
    \tikz\node[TTfamily]
            {\$ git archive --format=tar --prefix=foo/ -o foo.tar.gz master};\\[2em]
            
  Git also can generate an archive of packed objects and references to be imported into a repository (useful if machines are not directly connected), e.g.\\[1ex]
    \tikz\node[TTfamily]
            {[alice@earth \~{}]\$ git bundle create /tmp/foo-master.git master\\
            \# /tmp/foo-master.git is copied to moon by some means.\\[1ex]
            [bob@moon \~{}]\$ git clone -b master \~{}/foo-master.git\\[1ex]
            \# Or if the repository already exists\ldots{}\newline
            [bob@moon \~{}]\$ git remote add foo-bundle \~{}/foo-master.git\newline
            [bob@moon \~{}]\$ git pull foo-bundle master};
\end{frame}

\begin{frame}{ReReRe}
  ``git-rerere - Reuse recorded resolution of conflicted merges''
  \\[1em]
  \alert{FYI, see the \texttt{git-rerere(1)} manual page.}
\end{frame}
