% ==== [sudoer] MORE PORCELAIN ====
\section{\emph{[sudoer]} More porcelain}

\begin{frame}[fragile]{Save/Restore a dirty working directory}
  \texttt{git-stash(1)} saves the current state of the working directory + the index, and goes back to a clean WD.

  Saved changes can be restored with {\ttfamily\scriptsize\$ git stash pop}. Git-stash stack can be dumped by {\ttfamily\scriptsize\$ git stash list}.
  \pause

  \begin{lstlisting}[style=bash]
    $ echo foo > README.md
    $ git status
    On branch foo
    Changes not staged for commit:
    ``        `{\color{cred}modified:   README.md}\pause`

    $ git stash
    Saved working directory and index state WIP on foo: 9f7f586 README.md has been added
    $ git status
    On branch foo
    nothing to commit, working tree clean`\pause`

    $ git stash pop
    On branch foo
    Changes not staged for commit:
    ``        `{\color{cred}modified:   README.md}`
    Dropped refs/stash@{0} (35365e0c188e877ded1ecdd8190ec5bb1b6c2c1b)
  \end{lstlisting}
\end{frame}

\begin{frame}{Applying changes from other branches}
  \begin{columns}
    \column{.55\textwidth}
    \texttt{git-cherry-pick(1)} apply the changes introduced by the given commits, e.g.\par
    {\ttfamily\scriptsize\$ git cherry-pick 9f7f586}.\\[1em]
    The patch may not apply cleanly; if that is the case, you are required to resolve conflicts

    \column{.45\textwidth}
    \centering
    \tikz[every label/.style={font=\tiny}]{
      \graph[tree layout,grow'=up,nodes={simple commit obj,as=}]
            { a <- aa <- { b <- bb <- bbb[mLightBrown,label=left:9f7f586] <- bbbb[label=90:master],
                          c <- cc }
            };
      \draw<2>[<-] (cc) -- ++(0em,2.6em)
                   node(ccc)[simple commit obj,mLightBrown83,label=right:0c151e4]{};
      \foreach \i [count=\c] in {cc,ccc} \path<\c> node[above=0pt of \i,font=\tiny]{foo};
    }
  \end{columns}
\end{frame}

{
\tikzset{every label/.style={font=\tiny},
        commit 1/.style={simple commit obj,fill=teal},
        commit 2/.style={simple commit obj,fill=teal!74},
        commit 3/.style={simple commit obj,fill=teal!56}}
\begin{frame}{Rebase (+ interactive rebase!) (1/3)}
  Sometimes you fork a branch and it becomes outdated w.r.t. its parent. Quite probably, you would merge the parent branch.\par
  \begin{center}
    \tikz[hole/.style={densely dotted},
          merge/.style={simple commit obj,mLightBrown}]{
      \graph[tree layout,grow'=right,nodes={simple commit obj,as=}] {
        a <- aa[red] <- { [nudge down=3ex] b[label=above:A] <- bb[label=above:B] <- bbb[label=above:C] }
      };
      \draw<2->[<-] (aa) -- ++(3ex,3ex) node(c)[commit 1]{} -- ++(3ex,0ex) node(cc)[commit 2]{};
      \draw<2->[hole] (cc) -- ++(5ex,0ex) node(ccc)[commit 3]{};
      \draw<3->[<-] (bbb) -- ++(5ex,0ex) node(bbbb)[merge]{}
                    (ccc) -- (bbbb);
      \draw<4->[hole] (ccc) -- ++(10ex,0ex) node(cccc)[simple commit obj]{}
                    (bbbb) -- ++(10ex,0ex) node(bbbbb)[merge]{};
      \draw<4->[<-] (cccc) -- (bbbbb);
      % branch heads
      \foreach \i/\j/\k in {1/aa/master,1-2/bbb/topic,
                          2-3/ccc/master, 3/bbbb/topic,
                          4/cccc/master,4/bbbbb/topic}
        \path<\i> node[right=0pt of \j,font=\tiny]{\k};
    }
  \end{center}
  
  \onslide<5>{
    This clutters project history. Reapplying \texttt{topic} commits on top of \texttt{master} is better!\par
    \begin{center}
    \tikz\graph[tree layout,grow'=right,nodes={simple commit obj,as=}] {
        a <- aa[red] <- c[commit 1] <- cc[commit 2] <- ccc[commit 3,label=right:master]
            <- { [nudge down=3ex] b[label=above:A'] <- bb[label=above:B'] <- bbb[label=above:C',label=right:topic] }
      };
    \end{center}

    \tikz\node[TTfamily]
              {{\color{cyellow}\# Assuming that 'topic' is the current branch, this gives the result above}\\
              \$ git rebase master};
  }
\end{frame}

\begin{frame}{Rebase (+ interactive rebase!) (2/3)}
  It is one of the most powerful Git commands. In fact, it can be used to rewrite project history (next slide).\par\pause
  If there are conflicts, you will have to resolve them (as in merge).\pause

  \warning{\texttt{GIT-REBASE(1)} IMPLICATIONS:}
  \begin{itemize}
  \item Requires rewriting commits and is \alert{PROBLEMATIC} if you already pushed those objects
  \item You can break things: \alert{YOU HAVE BEEN WARNED!}
  \item If you ever force-push a rebased branch, others will have to fix their history. See git-rebase(1), section ``RECOVERING FROM UPSTREAM REBASE''.
  \end{itemize}
\end{frame}

\begin{frame}[fragile]{Rebase (+ interactive rebase!) (3/3)}
  \texttt{git-rebase(1)} has an interactive mode in which you can edit/reorder/remove the commits which are rebased.\\ It is very common to rewrite part of a branch to have a more meaningful history, e.g.

  \begin{center}
    \tikz\graph[tree layout,grow'=right,nodes={simple commit obj,as=}] {
        a <- aa[red] <- { c <- cc[label=right:master],
            b <- bb[label=below:\LT after-this-commit\GT]
              <- bbb[commit 1,label=above:A] <- bbbb[commit 2,label=above:B] <- bbbbb[commit 3,label=above:C,label=right:topic] }
      };
  \end{center}

  \begin{lstlisting}[style=bash]
    # This fires up an editor and gives you the chance to edit the commit list before they are applied (commits A, B and C)
    $ git rebase -i <after-this-commit>
  \end{lstlisting}

  \gitRebaseWarning
\end{frame}
}

\begin{frame}[fragile]{More about fixing history (1/2)}
  So common that \texttt{git-commit(1)} has the \texttt{--squash} and \texttt{--fixup} options. They mark commits to be automatically squashed. Rewriting occurs after a {\ttfamily\scriptsize\$ git rebase --autosquash}.
  \pause
  \begin{lstlisting}[style=bash]
    $ git log --oneline
    e7a2019 (HEAD -> master) Any other changes
    9f7f586 Added README.md
    02a7fb9 Added bar.txt`\pause`

    $ echo foo >> README.md && git commit -a --fixup 9f7f586
    $ git log --oneline
    24a54df (HEAD -> master) fixup! Added README.md
    e7a2019 Any other changes
    9f7f586 Added README.md
    02a7fb9 Added bar.txt`\pause`

    $ git rebase -i --autosquash 02a7fb9
    Successfully rebased and updated refs/heads/master.
    $ git log --oneline
    528efb7 (HEAD -> master) Any other changes
    a59735c Added README.md
    02a7fb9 Added bar.txt
  \end{lstlisting}

  \gitRebaseWarning
\end{frame}

\begin{frame}{More about fixing history (2/2)}
  If you only need to rewrite the last commit use\\ {\ttfamily\scriptsize\$ git commit --amend}

  \gitRebaseWarning
\end{frame}

\begin{frame}{git filter-branch}
  \Q I know how to rewrite commits. Can I automate the process?\\
  \pause
  \A \texttt{git-filter-branch(1)} lets you rewrite branches, applying filters to modify each tree/information about each commit, e.g. \\[1em]

  \tikz\node[TTfamily]
            {\$ git log --oneline\\
            {\color{cyellow}92cb761} (HEAD -\GT{} foo) Added nsswitch.conf\\
            {\color{cyellow}9f7f586} Added README.md\\
            {\color{cyellow}02a7fb9} (bar) Added bar.txt\\[1ex]
            \$ git filter-branch --msg-filter 'sed -e "s/Added \textbackslash([[:graph:]]*\textbackslash)\$/\textbackslash{}1 has been added/"' foo\\[1ex]
            \$ git log --oneline\\
            {\color{cyellow}6e9fbd6} (HEAD -\GT{} foo) nsswitch.conf has been added\\
            {\color{cyellow}63feb3c} README.md has been added\\
            {\color{cyellow}2fe54f3} bar.txt has been addded};

  \gitRebaseWarning
\end{frame}

\begin{frame}{Comparing branches}
  \Q Can I see where each of the given branches is w.r.t. others?\\
  \pause
  \A \texttt{git-show-branch} is your friend. Also, \texttt{git log --graph --oneline \ldots}\\[1em]

  \tikz\node[TTfamily]
            {\$ git show-branch master foo\\
            {\color{cred}!} [master] Added README.md\newline
            \phantom{!}{\color{cgreen}*} [foo] Added nsswitch.conf\newline
            \phantom{!*}{\color{cyellow}!} [bar] Added bar.txt\\
            ---\newline
            \phantom{+}{\color{cgreen}*}\phantom{+} [foo] Added nsswitch.conf\\
            {\color{cred}+}{\color{cgreen}*}\phantom{+} [master] Added README.md\\
            {\color{cred}+}{\color{cgreen}*}{\color{cyellow}+} [bar] Added bar.txt\\[1ex]
            \# To include all remote-tracking and local branches:\\
            \$ git show-branch --all};
\end{frame}

\begin{frame}{Share by other means}
  TAR or ZIP archives of a particular tree can be created by \texttt{git-archive(1)}, e.g.\\[1ex]
    \tikz\node[TTfamily]
            {\$ git archive --format=tar --prefix=foo/ -o foo.tar.gz master};\\[2em]
            
  Git also can generate an archive of packed objects and references to be imported into a repository (useful if machines are not directly connected), e.g.\\[1ex]
    \tikz\node[TTfamily]
            {[alice@earth \~{}]\$ git bundle create /tmp/foo-master.git master\\
            \# /tmp/foo-master.git is copied to moon by some means.\\[1ex]
            [bob@moon \~{}]\$ git clone -b master \~{}/foo-master.git\\[1ex]
            \# Or if the repository already exists\ldots{}\newline
            [bob@moon \~{}]\$ git remote add foo-bundle \~{}/foo-master.git\newline
            [bob@moon \~{}]\$ git pull foo-bundle master};
\end{frame}

\begin{frame}{ReReRe}
  ``git-rerere - Reuse recorded resolution of conflicted merges''
  \\[1em]
  \alert{FYI, see the \texttt{git-rerere(1)} manual page.}
\end{frame}
