% ==== GIT ESSENTIALS ====
\section{Git essentials}

\begin{frame}{Working tree and .git/ directory (1/2)}
  \begin{columns}
    \column{.55\textwidth}
    \begin{description}[]
    \item[\texttt{.git/} directory:] contains Git administrative and control files.
    \item[Working tree:] the tree of checked out files.
    \end{description}

    \column{.45\textwidth}
    \centering
    \begin{forest} for tree={directory tree}
      [repository/
        [.git/
          [config] [HEAD] [\ldots]
        ]
        [Makefile] [main.cpp]
        [\ldots]
      ]
    \end{forest}
  \end{columns}
\end{frame}

\begin{frame}{Working tree and .git/ directory (2/2)}
  \begin{columns}
    \column{.55\textwidth}
    \begin{description}[]
    \item[Bare repository:] \alert{NO} working tree + \alert{NO} \texttt{.git/} directory sub-directory.\par
      Git files directly present in the directory.
    \end{description}

    \column{.45\textwidth}
    \centering
    \begin{forest} for tree={directory tree}
      [repository.git/
        [config] [HEAD] [\ldots]
      ]
    \end{forest}
  \end{columns}
\end{frame}

\begin{frame}{Objects, references and symrefs (1/3)}
  \begin{description}[]
  \item<1->[Object:] raw octets stored in Git; identified by its SHA-1.\par
    Types: \emph{commit}, \emph{tree}, \emph{blob}, \emph{tag}.
  \item<2->[Ref(erence):] a name that points to an object. Hierarchical namespace rooted at \texttt{refs/}
  \item<3->[Symref:] a ref that points to another ref, e.g. HEAD.
  \end{description}

  \begin{center}
    \only<3->{\tikz[every node/.style={font=\tiny,on chain,join},
      every join/.style={->},start chain=going below,node distance=20pt] \path
        node [symref] {HEAD}
        node        {refs/heads/master}
        node [draw] {770fcfa540f5f5d1f49570b9d09320c7a7b7e879 (commit)};}
  \end{center}
\end{frame}

\begin{frame}{Objects, references and symrefs (2/3)}
  \begin{columns}
    \column{.5\textwidth}
    The contents of an object depend on its type:
    \begin{description}[]
    \item<1->[Blob:] raw data; stores file contents.
    \item<2->[Tree:] directory contents.
    \item<3->[Commit:] information about a revision.
    \item<4->[Tag:] ref pointing to a commit + message + PGP signature (optional).
    \end{description}

    \column{.5\textwidth}
    \centering
    \tikz[every label/.style={font=\tiny,rotate=0}]{
      % blob objects
      \node[simple blob obj={9355a87}] {\#\\ \# /etc/hosts: \ldots{}};
      \node[simple blob obj={0632e41},below=24pt of 9355a87] {\# Generated by NetworkManager\\ search arcos.\ldots{}};
      % tree
      \draw<2->[->] node[simple tree obj={3814f01}{9355a87/hosts,
                                        0632e41/resolv.conf},
        below=12pt of 9355a87,xshift=-70pt]{};
      % commit chain
      \draw<3->[->] graph[tree layout,grow'=up,nodes={simple commit obj,as=}] {
        a <- aa <- aaa[label={left:efa21b0},desired at={(-90pt,-23pt)}] <- aaaa;
      }
      (aaaa) |- (3814f01);
      % tag
      \draw<4->[<-] (aa) -- ++(-30pt,0pt) node[simple tag obj={a27c52b}]{};
    }
  \end{columns}
\end{frame}

\begin{frame}{Objects, references and symrefs (3/3)}
  Typically, objects can be reached given a ref (but not always).
  \begin{columns}
    \column{.5\textwidth}
    \begin{center}
      \tikz\graph[tree layout,grow'=up,nodes={simple commit obj,as=}]{
        a -- { b -- c[red],
          d -- e -- f[label={north:refs/heads/master}] };
        };
    \end{center}\par
    \textbf{Unreachable object:} an object which is not reachable from any reference.

    \column{.5\textwidth}
    \vrule height .117\textheight width0pt
    \begin{center}
      \tikz\graph[tree layout,grow'=up,nodes={simple commit obj,as=}]{
        a -- b -- c[label={north:refs/heads/master}];
        d[red];
        };
    \end{center}\par
    \textbf{Dangling object:} not reachable even from other unrechable objects.
  \end{columns}
  
  \hfill {\tiny More at \texttt{gitglossary(7)}}
\end{frame}

\begin{frame}{Project history, branches and tags}
  \begin{columns}
    \column{.67\textwidth}
    \only<1->{Commit objects form a DAG (they point to their parents). This DAG is known as the history of a project.\\[-.6em] \rule{1in}{.1pt}\\[.3em]}
    \only<2-4>{\textbf{Branch:} an active line of development; \emph{tip:} the most recent commit.\\[.5em]}
    \only<3-4>{\textbf{(Branch) head:} a reference to the tip of a branch.\par Local heads at: \texttt{refs/heads/}.\\[.5em]}
    \only<4>{\textbf{Remote-tracking branch:} a ref to a remote head; follow changes from another repository.\par At \texttt{refs/remotes/*/}.}

    \only<5>{\textbf{Merge commit:} a commit object that has $\geq 2$ parents.\\[.5em]}
    \only<5>{\textbf{Octopus:} a merge that has $> 2$ parents.}

    \column{.33\textwidth}
    \tikz\graph[layered layout,grow'=down,level distance=2em,
              nodes={simple commit obj,as=}]{
      a <- aa[mLightBrown] <- { b <- bb <- bbb[label={70:refs/heads/foo}],
                                c <- cc[label={135:refs/remotes/origin/master}] }
        <- aaa[red] <- aaaa[label={north:refs/heads/master}];
    };
  \end{columns}
\end{frame}

\begin{frame}{The ``index'' (cache) file}
  \textbf{Short story:} basically, it is the staging area for the next commit.\\[1em]

  \begin{itemize}[<+->]
  \item ``A collection of files with stat information, whose contents are stored as objects.'' [\texttt{gitglossary(7)}]
  \item For each file, it stores \LT object SHA-1\GT\ \LT attributes\footnote{Last modified time, size, etc.}\GT\\
    \tikz\node[TTfamily]
              {100644 01cb7066623241a0e5714a6630f0355eb0c80de4 0\ \ \ \ .gitignore\\
                \ldots{}\\
                100644 94fbec4cf383e9122c22d60cfad91b3c897e2c63 0\ \ \ \ slides.tex};
  \item Changes to the working tree found by comparing these attributes.
  \item Entries may be updated (\texttt{git add}) and new commits may be created from the index.
  \end{itemize}
\end{frame}

\begin{frame}{Other definitions (1/3)}
  \textbf{Fast-forward:} a special type of merge; given two heads $A$ and $B$, merging $B$ into $A$ is considered fast-forward if $merge\_base(A, B) == A$, i.e. $A$ is ancestor of $B$.\\[1em]
  \begin{columns}
    \tikzset{every label/.append style={text width=}}
    \column{.5\textwidth}
    \begin{center}
      \tikz\graph[tree layout,grow'=up,level distance=7pt,
        nodes={simple commit obj,as=}] {
      a <- aa[red,label={north:A}] <-
      {[nudge right=3ex]
        c <- cc <- ccc[label={north:B}] };
      };
    \end{center}\par
    Fast-forward (update ref only!)

    \column{.5\textwidth}
    \vrule height .067\textheight width0pt
    \begin{center}
      \tikz\graph[tree layout,grow'=up,level distance=7pt,
        nodes={simple commit obj,as=}] {
      a <- aa[red] <-
      { b[label={north:A}],
        c <- cc <- ccc[label={north:B}] }; 
      };
    \end{center}\par
    Non fast-forward (requires a merge)
  \end{columns}
\end{frame}

\begin{frame}{Other definitions (2/3)}
  \begin{columns}
    \tikzset{every label/.append style={text width=}}
    \column{.5\textwidth}
    \begin{description}[]
    \item<1-5>[HEAD:] symref that dereferences to the current checked-out head.
    \item<2-5>[Detached HEAD:] HEAD may also point at an arbitrary commit, i.e. ``detached'' from any branch. You may make commits in this state, but\ldots{}
    \end{description}
    \only<6>{if HEAD is made to point somewhere else, they will become unreachable (and eventually deleted by the GC). \alert{Create a ref to avoid this!} }
    
    \column{.5\textwidth}
    \centering
    \tikz{
      \graph[tree layout,grow'=up,nodes={simple commit obj,as=}] {
      a <- aa[label={135:c864ac8}] <- aaa <- aaaa[label={north:refs/heads/master}];
      };
      % commit chain; head detached at c864ac8
      {[simple commit obj/.append style={at end}]
        \draw<3->[<-] (aa) -- ++(-3em,2em) node (b)[simple commit obj]  {};
        \draw<4->[<-] (b)  -- ++(0ex,2em)  node (bb)[simple commit obj] {};
        \draw<5->[<-] (bb) -- ++(0ex,2em)  node (bbb)[simple commit obj]{};
      }
      
      \foreach \i [count=\c] in {aaaa,aa,b,bb,bbb,aaaa}
        \draw<\c>[<-] ([xshift=-14pt,yshift=1pt]\i.north) to [bend left] node[symref,at end,above] {HEAD} ++(-10pt,15pt);
    }
  \end{columns}
\end{frame}

\begin{frame}{Other definitions (3/3)}
  \textbf{Reflog:} stores the local history of a ref.\\[1em]

  \begin{itemize}
  \item What was \texttt{HEAD} pointing at before the last change?
  \item What did \texttt{refs/heads/foo} pointed at two weeks ago?
  \end{itemize}
\end{frame}
